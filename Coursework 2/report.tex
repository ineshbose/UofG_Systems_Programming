% SP(H) CW2 Report written in LaTeX by Inesh Bose

\documentclass{article}

%==============================================================================
%% Packages and Command Setting
\usepackage[utf8]{inputenc}
\setcounter{secnumdepth}{0}
\usepackage[shortlabels]{enumitem}
\usepackage{layout}
\usepackage{multirow}
\usepackage{hyperref}
\usepackage{minted}
\newcommand{\code}[1]{\texttt{#1}}
\newcommand{\entry}[3]{\begin{tabular}[c]{@{}l@{}}\code{#1}\\ \code{#2}\\ \code{#3}\end{tabular}}
 
%==============================================================================
%% Title Page

\title{Systems Programming\\ \normalsize{\textbf{Coursework 2:} Concurrent Dependency Discoverer}}
\author{\bf Inesh Bose}
\date{}

\begin{document}

\definecolor{bg}{rgb}{0.95,0.95,0.95}

\maketitle
\vspace{1cm}

\tableofcontents
\addtocontents{toc}{\protect\thispagestyle{empty}}
\pagenumbering{gobble}

%==============================================================================
%% Status Report

\vspace{2cm}

\section{Status Report}

The program compiles and runs successfully. No bugs or limitations have been found so far. No special changes need to be made to the environment (except making sure that you run \code{source /usr/local/bin/clang9.setup} on uni-machines for C++17). \code{dependencyDiscoverer.cpp} is the only source-code file that is mostly the same but has been refactored according to the specification that enables successful threading according to number of threads given in \code{CRAWLER\_THREADS} that can be done using \code{EXPORT CRAWLER\_THREADS=4}.

\vspace{1cm}

%==============================================================================
%% Screenshots: Build & Runtimes

\newpage

\section{Screenshots: Build \& Runtimes}

\vspace{0.2cm}

The following screenshots are in the form of \code{vertabim}/\code{minted} blocks instead and have not be falsified.

\vspace{0.5cm}

\begin{enumerate}[a)]

    \item The path of executing the program (\code{pwd})
\begin{minted}[bgcolor=bg]{bash}
-bash-4.2$ pwd
/users/level3/2504266b/systems_programming/Coursework 2
\end{minted}

\vspace{0.5cm}

    \item Crawler being compiled either using manually by a sequence of commands or by a Makefile
        \begin{minted}[bgcolor=bg]{bash}
-bash-4.2$ make
clang++ -Wall -Werror -std=c++17 -o dependencyDiscoverer
    dependencyDiscoverer.cpp -lpthread
        \end{minted}

\vspace{0.5cm}

    \item The time to run the sequential crawler on all \code{.c}, \code{.l} and \code{.y} files in the \code{test} directory
        \begin{minted}[bgcolor=bg]{bash}
-bash-4.2$ cd test
-bash-4.2$ time ../dependencyDiscoverer *.y *.l *.c
real    0m0.087s
user    0m0.014s
sys     0m0.016s
        \end{minted}

\vspace{0.5cm}

    \item the time to run the threaded crawler with one thread
        \begin{minted}[bgcolor=bg]{bash}
-bash-4.2$ export CRAWLER_THREADS=1
-bash-4.2$ time ../dependencyDiscoverer *.y *.l *.c
real    0m0.053s
user    0m0.010s
sys     0m0.015s
        \end{minted}

\end{enumerate}

\vspace{0.5cm}

%==============================================================================
%% Runtime with Multiple Threads : Part 1

\newpage

\section{Runtime with Multiple Threads}

\subsection{Screenshot}

\begin{minted}[bgcolor=bg]{bash}
-bash-4.2$ pwd
/users/level3/2504266b/systems_programming/Coursework 2/test

-bash-4.2$ export CRAWLER_THREADS=1
-bash-4.2$ time ../dependencyDiscoverer *.y *.l *.c
real    0m0.052s
user    0m0.019s
sys     0m0.011s

-bash-4.2$ export CRAWLER_THREADS=2
-bash-4.2$ time ../dependencyDiscoverer *.y *.l *.c
real    0m0.027s
user    0m0.008s
sys     0m0.018s

-bash-4.2$ export CRAWLER_THREADS=3
-bash-4.2$ time ../dependencyDiscoverer *.y *.l *.c
real    0m0.022s
user    0m0.009s
sys     0m0.018s

-bash-4.2$ export CRAWLER_THREADS=4
-bash-4.2$ time ../dependencyDiscoverer *.y *.l *.c
real    0m0.021s
user    0m0.015s
sys     0m0.014s

-bash-4.2$ export CRAWLER_THREADS=6
-bash-4.2$ time ../dependencyDiscoverer *.y *.l *.c
real    0m0.019s
user    0m0.012s
sys     0m0.020s

-bash-4.2$ export CRAWLER_THREADS=8
-bash-4.2$ time ../dependencyDiscoverer *.y *.l *.c
real    0m0.019s
user    0m0.018s
sys     0m0.017s
\end{minted}

%==============================================================================
%% Runtime with Multiple Threads : Part 2

\newpage

\subsection{Experiment}

\begin{table}[htb]
\centering
\begin{tabular}{|l|l|l|l|}
\hline
\multirow{2}{*}{\begin{tabular}[c]{@{}l@{}}CRAWLER\_\\ THREADS\end{tabular}} &
  \multicolumn{1}{c|}{1} &
  \multicolumn{1}{c|}{2} &
  \multicolumn{1}{c|}{3} \\ \cline{2-4} 
 &
  \multicolumn{3}{c|}{Elapsed Time} \\ \hline
Execution 1 &
  \entry{real 0m0.042s}{user 0m0.009s}{sys 0m0.015s} &
  \entry{real 0m0.027s}{user 0m0.011s}{sys 0m0.015s} &
  \entry{real 0m0.022s}{user 0m0.012s}{sys 0m0.015s} \\ \hline
Execution 2 &
  \entry{real 0m0.042s}{user 0m0.015s}{sys 0m0.009s} &
  \entry{real 0m0.028s}{user 0m0.009s}{sys 0m0.017s} &
  \entry{real 0m0.023s}{user 0m0.009s}{sys 0m0.017s} \\ \hline
Execution 3 &
  \entry{real 0m0.087s}{user 0m0.012s}{sys 0m0.026s} &
  \entry{real 0m0.028s}{user 0m0.016s}{sys 0m0.011s} &
  \entry{real 0m0.023s}{user 0m0.013s}{sys 0m0.014s} \\ \hline
Median &
  \entry{real 0m0.042s}{user 0m0.012s}{sys 0m0.015s} &
  \entry{real 0m0.028s}{user 0m0.011s}{sys 0m0.015s} &
  \entry{real 0m0.023s}{user 0m0.012s}{sys 0m0.015s} \\ \hline
\end{tabular}
\end{table}

\vspace{-0.5cm}

\begin{table}[htb]
\centering
\begin{tabular}{|l|l|l|l|}
\hline
\multirow{2}{*}{\begin{tabular}[c]{@{}l@{}}CRAWLER\_\\ THREADS\end{tabular}} &
  \multicolumn{1}{c|}{4} &
  \multicolumn{1}{c|}{6} &
  \multicolumn{1}{c|}{8} \\ \cline{2-4} 
 &
  \multicolumn{3}{c|}{Elapsed Time} \\ \hline
Execution 1 &
  \entry{real 0m0.020s}{user 0m0.012s}{sys 0m0.016s} &
  \entry{real 0m0.019s}{user 0m0.015s}{sys 0m0.016s} &
  \entry{real 0m0.018s}{user 0m0.012s}{sys 0m0.019s} \\ \hline
Execution 2 &
  \entry{real 0m0.020s}{user 0m0.012s}{sys 0m0.016s} &
  \entry{real 0m0.020s}{user 0m0.013s}{sys 0m0.019s} &
  \entry{real 0m0.019s}{user 0m0.010s}{sys 0m0.024s} \\ \hline
Execution 3 &
  \entry{real 0m0.021s}{user 0m0.013s}{sys 0m0.015s} &
  \entry{real 0m0.020s}{user 0m0.013s}{sys 0m0.019s} &
  \entry{real 0m0.020s}{user 0m0.012s}{sys 0m0.023s} \\ \hline
Median &
  \entry{real 0m0.020s}{user 0m0.012s}{sys 0m0.016s} &
  \entry{real 0m0.020s}{user 0m0.013s}{sys 0m0.019s} &
  \entry{real 0m0.019s}{user 0m0.012s}{sys 0m0.023s} \\ \hline
\end{tabular}
\end{table}


%==============================================================================
%% Runtime with Multiple Threads : Part 3

\vspace{0.5cm}

\subsection{Discussion}

With threading, we clearly see how the execution time (\code{real} time mainly) decreases drastically. Specially when shifting from one thread to two which splits the processing. Since these times are not for a \textbf{\textit{relatively} huge} program, the decrease in execution time from two threads to eight is not significant but is still there. \code{user} time is mostly constant, and \code{sys} time is variable.

\end{document}
